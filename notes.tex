\documentclass[letterpaper]{article}
\usepackage{amsmath}
\usepackage{amsthm}
\usepackage{amssymb}

\title{Notes on Order, Well-Order, and Cardinality}
\author{Mike Haskel}
\date{Fall, 2014}

\newtheorem{theorem}{Theorem}[section]

\theoremstyle{definition}
\newtheorem{definition}[theorem]{Definition}
\newtheorem{exercise}[theorem]{Exercize}
\newtheorem{example}[theorem]{Example}

\begin{document}
\maketitle

\section{Introduction}
The purpose of these notes is to rigorously introduce a handful of
related core concepts that permeate mathematics.  The intended
audience is any undergraduate or accellerated high school student who
is familiar with the basic notation and concepts surrounding sets and
functions.  The student should also be somewhat comfortable with the
language involved in making and proving precise mathematical
statements; they should, for example, be able to explain the
difference between the sentences
\begin{quote}
  for all $x$, there is a $y$ such that $x+y=0$
\end{quote}
and
\begin{quote}
  there is $y$ such that, for all $x$, $x+y=0$.
\end{quote}

These notes will introduce the concepts of order, well-order, and
cardinality.  Order is what it sounds like: any way of comparing two
things that behaves as a comparison of size.  We draw attention to
some of the different kinds of orders and the axioms (rules) they
obey.  A particularly important type of order is a \emph{well-order}.
Well orders behave like the natural numbers in that we can perform a
certain kind of induction them.  We then use the ideas surrounding
well-orders to prove \emph{Zorn's Theorem}, a broad and important
mathematical tool.  With all these concepts in our arsenal, we finally
turn to introduce the idea of \emph{cardinality}: the size of a
(possibly infinite) set.

\section{Background Review: Sets, Relations, and Functions}
TODO.

\section{Order}
We want to study the idea of an order: a binary relation that behaves
as a comparison of size.  We approach this idea in three steps, each
of which is a useful concept in its own right.

\begin{definition}
  \label{def:preorder}
  Let \(X\) be a set and \(\leq\) a binary relation on \(X\).  Then
  \(\leq\) is a \emph{preorder} on \(X\) if it is reflexive and
  transitive.
\end{definition}

\begin{definition}
  \label{def:partialorder}
  Let \(X\) be a set and \(\leq\) a preorder on \(X\).  Then \(\leq\)
  is a \emph{partial order} if it is also antisymmetric.
\end{definition}

Note that the only defect of a preorder that prevents it from being a
partial order is that it may be possible to have two elements \(a,b\)
in \(X\) with \(a \leq b\) and \(b \leq a\), but nevertheless \(a \neq
b\).  When \(\leq\) is a partial order, it is common to write \(a \sim
b\) to denote that \(a \leq b\) and \(b \leq a\).  We can also use \(a
\geq b\) as obvious shorthand for \(b \leq a\), and \(a < b\) as
shorthand for indicating simultaneously that \(a \leq b\) and \(a
\not\sim b\).

\begin{example}
  \label{example:prenotpartial}
  Consider the set
  \[X = \{\text{black}, \text{red}, \text{blue}\}\text{,}\]
  and the binary relation
  \begin{align*}
    \mathord{\leq} = \{
    &(\text{black}, \text{black}), \\
    &(\text{black}, \text{red}), \\
    &(\text{black}, \text{blue}), \\
    &(\text{red}, \text{red}), \\
    &(\text{red}, \text{blue}), \\
    &(\text{blue}, \text{red}), \\
    &(\text{blue}, \text{blue})
    \} \text{.}
  \end{align*}
  It is routine to check that \(\leq\) is a preorder.  It is not a
  partial order, however, because we have \(\text{red} \sim
  \text{blue}\); that is, \(\text{red} \leq \text{blue}\) and
  \(\text{blue} \leq \text{red}\).
\end{example}

\begin{example}
  \label{example:partialnotlinear}
  Consider the set
  \[X = \{\text{black}, \text{red}, \text{blue}\}\text{,}\]
  and the binary relation
  \begin{align*}
    \mathord{\leq} = \{
    &(\text{black}, \text{black}), \\
    &(\text{black}, \text{red}), \\
    &(\text{black}, \text{blue}), \\
    &(\text{red}, \text{red}), \\
    &(\text{blue}, \text{blue})
    \} \text{.}
  \end{align*}
  Again, it is routine to check that \(\leq\) is a preorder.  With our
  modification, however, \(\leq\) is also a partial order.  Note in
  particular the new behavior of red and blue: neither one is \(\leq\)
  the other.  We say that red and blue are \emph{incomparable}.
\end{example}
\end{document}
