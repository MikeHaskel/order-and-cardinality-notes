\documentclass[letterpaper]{article}
\usepackage{amsmath}
\usepackage{amsthm}
\usepackage{amssymb}

\title{Notes on Order, Well-Order, and Cardinality}
\author{Mike Haskel}
\date{Fall, 2014}

\begin{document}
\maketitle

\section{Introduction}
The purpose of these notes is to rigorously introduce a handful of
related core concepts that permeate mathematics.  The intended
audience is any undergraduate or accellerated high school student who
is familiar with the basic notation and concepts surrounding sets and
functions.  The student should also be somewhat comfortable with the
language involved in making and proving precise mathematical
statements; they should, for example, be able to explain the
difference between the sentences
\begin{quote}
  for all $x$, there is a $y$ such that $x+y=0$
\end{quote}
and
\begin{quote}
  there is $y$ such that, for all $x$, $x+y=0$.
\end{quote}

These notes will introduce the concepts of order, well-order, and
cardinality.  Order is what it sounds like: any way of comparing two
things that behaves as a comparison of size.  We draw attention to
some of the different kinds of orders and the axioms (rules) they
obey.  A particularly important type of order is a \emph{well-order}.
Well orders behave like the natural numbers in that we can perform a
certain kind of induction them.  We then use the ideas surrounding
well-orders to prove \emph{Zorn's Theorem}, a broad and important
mathematical tool.  With all these concepts in our arsenal, we finally
turn to introduce the idea of \emph{cardinality}: the size of a
(possibly infinite) set.

\end{document}
