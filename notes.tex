\documentclass[letterpaper]{article}
\usepackage{amsmath}
\usepackage{amsthm}
\usepackage{amssymb}

\title{Notes on Order, Well-Order, and Cardinality}
\author{Mike Haskel}
\date{Fall, 2014}

\newtheorem{theorem}{Theorem}[section]
\newtheorem{corollary}{Corollary}[theorem]
\newtheorem{proposition}[theorem]{Proposition}
\newtheorem{lemma}[theorem]{Lemma}

\theoremstyle{definition}
\newtheorem{definition}[theorem]{Definition}
\newtheorem{exercise}[theorem]{Exercise}
\newtheorem{example}[theorem]{Example}

\newcommand{\defterm}{\emph}
\renewcommand{\subset}{\subseteq}
\newcommand{\powerset}{\mathcal{P}}
\newcommand{\cartesianprod}{\times}
\newcommand{\inverse}[1]{{#1^{-1}}}
\newcommand{\compose}{\circ}
\renewcommand{\phi}{\varphi}

\begin{document}
\maketitle

\section{Introduction}
The purpose of these notes is to rigorously introduce a handful of
related core concepts that permeate mathematics.  The intended
audience is any undergraduate or accellerated high school student who
is familiar with the basic notation and concepts surrounding sets and
functions.  The student should also be somewhat comfortable with the
language involved in making and proving precise mathematical
statements; they should, for example, be able to explain the
difference between the sentences
\begin{quote}
  for all $x$, there is a $y$ such that $x+y=0$
\end{quote}
and
\begin{quote}
  there is $y$ such that, for all $x$, $x+y=0$.
\end{quote}

\section{Sets, Functions, and Relations}

\begin{definition}
  A \defterm{set} is any collection of elements.  If \(X\) is a set,
  we write \(x \in X\) to denote that \(x\) is an element of \(X\).
  Two sets are considered equal if they have the same elements.
  \footnote{We will sometimes refer to a collection of elements as a
    \defterm{class}.  The distinction between a set and a class is
    subtle and necessary only to avoid certain technical
    contradictions, but for most purposes sets and classes behave
    identically; in particular, almost all definitions and theorems we
    give for sets work just as well for classes.  The main practical
    distinctions between the two concepts are that
    \begin{itemize}
      \item we can usually only form a set using rules based on some
        other set (e.g., if \(X\) is a set, the collection of subsets
        \(X\) is a set), while we can form a class based on more
        arbitrary rules (e.g, the collection of all finite sets is a
        class); and
      \item we allow sets to be elements of other collections, but we
        do not consider collections whose elements are classes.
    \end{itemize}}
\end{definition}

\begin{definition}
  Let \(X\) and \(Y\) be sets.  We say that \(X\) is a
  \defterm{subset} of \(Y\), and write \(X \subset Y\), if every
  element of \(X\) is also an element of \(Y\).
\end{definition}

\begin{definition}
  Let \(x\) and \(y\) be anything.  By \((x,y)\) we denote the
  \defterm{ordered pair} consisting of \(x\) followed by \(y\).  Given
  two ordered pairs \((x_0,y_0)\) and \((x_1,y_1)\), we consider
  \((x_0,y_0) = (x_1,y_1)\) if \(x_0 = x_1\) and \(y_0 = y_1\).
\end{definition}

\begin{definition}
  Let \(X\) and \(Y\) be sets.  The \defterm{Cartesian product} of
  \(X\) and \(Y\), written \(X \cartesianprod Y\), is the set of all
  ordered pairs \((x,y)\) such that \(x \in X\) and \(y \in Y\).
\end{definition}

\begin{definition}
  Let \(X\) and \(Y\) be sets.  A \defterm{binary relation} from \(X\)
  to \(Y\) is a subset of \(X \cartesianprod Y\).  If \(X = Y\), we
  say that we have a binary relation on \(X\).  If \(R\) is a binary
  relation from \(X\) to \(Y\), \(x \in X\), and \(y \in Y\), we write
  either \(R(x,y)\) or \(xRy\) as shorthand for \((x,y) \in R\).
\end{definition}

\begin{definition}
  Let \(X\) be a set.  By the \defterm{power set} of \(X\), denoted
  \(\powerset(X)\), we mean the set of subsets of \(X\).  That is,
  \[A \in \powerset(X) \iff A \subset X \text{.}\]
\end{definition}

\subsection{Relations Between Sets}
\begin{definition}
  Let \(R\) be a binary relation from \(X\) to \(Y\).  \(R\) is
  \defterm{left total} if, for all \(x \in X\), there is at least one
  \(y \in Y\) such that \(xRy\).  \(R\) is \defterm{right total} if,
  for all \(y \in Y\), there is at least one \(x \in X\) such that
  \(xRy\).
\end{definition}

\begin{definition}
  Let \(R\) be a binary relation from \(X\) to \(Y\).  \(R\) is
  \defterm{right unique} if, for all \(x \in X\), there is at most one
  \(y \in Y\) such that \(xRy\).  \(R\) is \defterm{left unique} if,
  for all \(y \in Y\), there is at most one \(x \in X\) such that
  \(xRy\).  (Beware the unfortunately asymmetric use of left and right
  in the definitions of total and unique.)
\end{definition}

If \(y\) is the unique element such that \(x R y\), we often write
\(R(x)\) to denote \(y\).  Similarly, if \(x\) is the unique element
such that \(xRy\), we often write \(\inverse{R}(y)\) to denote \(x\).

\begin{definition}
  A relation which is left total and right unique is called a
  \defterm{function}.  A function which is also left unique is called
  an \defterm{injection} (or is said to be \defterm{injective}).  A
  function which is also right total is called a \defterm{surjection}
  (or is said to be \defterm{surjective}).  A function which is both
  an injection and a surjection (so it is left total, right total,
  left unique, and right unique) is called a \defterm{bijection} (or
  is said to be \defterm{bijective}).
\end{definition}

We often indicate that \(f\) is a function from \(X\) to \(Y\), by
writing \(f:X \to Y\).

\begin{definition}
  Let \(X\) be a set.  By \(id_X\), we mean the binary relation on
  \(X\) defined by
  \[id_X(x,y) \iff x = y \text{.}\]
\end{definition}

\begin{definition}
  When \(X \subset Y\), we can view \(id_X\) as an injection from
  \(X\) to \(Y\).  When viewed as such, this injection is called the
  \defterm{inclusion map} from \(X\) to \(Y\).
\end{definition}

\begin{definition}
  Let \(X\) and \(Y\) be sets, and let \(R\) be a binary relation from
  \(X\) to \(Y\).  By \(\inverse{R}\), we mean the binary relation
  from \(Y\) to \(X\) defined by
  \[\inverse{R}(y,x) \iff R(x,y) \text{.}\]
\end{definition}

\begin{definition}
  Let \(X\), \(Y\), and \(Z\) be sets, let \(R\) be a binary relation
  from \(X\) to \(Y\), and let \(S\) be a binary relation from \(Y\)
  to \(Z\).  By \(S \compose R\), we mean the binary relation from
  \(X\) to \(Z\) defined by
  \[S \compose R(x,z) \iff \text{ there is } y \in Y \text{ such that } R(x,y) \text{ and } S(y,z) \text{.}\]
\end{definition}

\begin{exercise}
  Show the following basic facts.
  \begin{enumerate}
    \item \(id_X\) is a bijection.
    \item \(\inverse{(\inverse{R})} = R\).
    \item \(\inverse{R}\) is left/right unique/total if and only if
      \(R\) is right/left unique/total.  Note in particular that the
      inverse of a function is a function if and only if both are
      bijections.
    \item If \(R\) is a binary relation from \(X\) to \(Y\), \(id_Y
      \compose R = R = R \compose id_X\).
    \item Composition is associative: \((R \compose S) \compose T = R
      \compose (S \compose T)\).
    \item \(\inverse{(R \compose S)} = \inverse{S} \compose
      \inverse{R}\).
    \item Let \(R\) be a binary relation from \(X\) to \(Y\).  If
      \(R\) is left total and left unique, then \(\inverse{R} \compose
      R = id_X\).  If \(R\) is right total and right unique, then \(R
      \compose \inverse{R} = id_Y\).
    \item If \(R\) and \(S\) are both left/right total/unique, then
      \(R \compose S\) is left/right total/unique.
  \end{enumerate}
\end{exercise}

\subsection{Relations On Sets}
\begin{definition}
  Let \(R\) be a binary relation on \(X\).  \(R\) is
  \defterm{reflexive} if, for all \(x \in X\), \(xRx\).
\end{definition}

\begin{definition}
  Let \(R\) be a binary relation on \(X\).  \(R\) is
  \defterm{transitive} if whenever \(xRy\) and \(yRz\), we
  additionally have \(xRz\).  When this occurs, we write \(xRyRz\) as
  shorthand.
\end{definition}

\begin{definition}
  Let \(R\) be a binary relation on \(X\).  \(R\) is
  \defterm{symmetric} if whenever \(xRy\), we additionally have
  \(yRx\).
\end{definition}

\begin{definition}
  Let \(R\) be a binary relation on \(X\).  \(R\) is
  \defterm{antisymmetric} if whenever \(xRy\) and \(yRx\), we
  additionally have \(x = y\).
\end{definition}

\begin{definition}
  A \defterm{preorder} on \(X\) is a binary relation on \(X\) which is
  reflexive and transitive.
\end{definition}

\begin{definition}
  An \defterm{equivalence relation} on \(X\) is a preorder on \(X\)
  which is additionally symmetric.
\end{definition}

\begin{definition}
  A \defterm{partial order} on \(X\) is a preorder on \(X\) which is
  additionally antisymmetric.
\end{definition}

\begin{definition}
  A \defterm{total order} on \(X\) is a partial order \(\leq\) on
  \(X\) with the additional property that, for all \(x,y \in X\),
  either \(x \leq y\) or \(y \leq x\).
\end{definition}

\begin{exercise}
  Let \(X\) be a set.  Show that \(id_X\) is an equivalence relation
  on \(X\).
\end{exercise}

\begin{exercise}
  Let \(X\) be a set.  Show that the subset relation is a partial
  order on \(\powerset(X)\).
\end{exercise}

Observe that the usual order on the real numbers is a total order.

\begin{definition}
  Let \(X\) be a set, and \(E\) be an equivalence relation on \(X\).
  A subset \(A\) of \(X\) is called an \(E\) \defterm{equivalence
    class} if there is some \(a \in A\) such that, for all \(x \in
  X\), \(aEx\) if and only if \(x \in A\).
\end{definition}

\begin{exercise}
  Let \(X\) be a set, \(E\) be an equivalence relation on \(X\), \(A
  \subset X\) be an \(E\) equivalence class, and \(a \in A\).  Then,
  for all \(x \in X\), \(aEx\) if and only if \(x \in A\).  That is,
  for any equivalence class, any element of that class can serve as
  the element used in the definition.
\end{exercise}

\begin{definition}
  Let \(X\) be a set, and \(E\) be an equivalence relation on \(X\).
  The \defterm{quotient} of \(X\) by \(E\), denoted \(X/E\), is the
  set of \(E\) equivalence classes.  That is,
  \[A \in X/E \iff A \text{ is an } E \text{ equivalence class.}\]
\end{definition}

\begin{exercise}
  Let \(X\) be a set, \(E\) be an equivalence relation on \(X\), and
  \(x \in X\).  Show that there is a unique \(E\) equivalence class
  containing \(x\), which we will denote \([x]_E\).  Note that this
  defines a surjection from \(X\) to \(X/E\), called the
  \defterm{quotient map}.
\end{exercise}

\section{Ordered Sets}
In this section, we will develop some more detailed machinery for
working with sets equipped with an order relation.  The main mechanism
for this machinery is that of a homomorphism.

\subsection{Homomorphisms}
A homomorphism is a function between sets that preserves whatever
additional structure those sets may be equipped with.  They are useful
in that they allow us to move information about ordering from one set
to another.

\begin{definition}
  Let \(X\) and \(Y\) be sets equipped with binary relations \(R_X\)
  and \(R_Y\), respectively.  A \defterm{homomorphism} from \(X\) to
  \(Y\) is a function \(f\) from \(X\) to \(Y\) with the property
  that, for all \(x_0,x_1 \in X\), \(x_0R_Xx_1\) if and only if
  \(f(x_0)R_Yf(x_1)\).
\end{definition}

\begin{exercise}
  Let \(X\) and \(Y\) be sets equipped with binary relations \(R_X\)
  and \(R_Y\), respectively, and let \(f:X \to Y\) be a homomorphism.
  If \(R_X\) is antisymmetric and \(R_Y\) is reflexive, then \(f\) is
  an injection.
\end{exercise}

\begin{definition}
  Let \(X\) be a set, let \(Y\) be a set equipped with a binary
  relation \(R\), and let \(f:X \to Y\) be a function.  The
  \defterm{pullback} of \(R\) along \(X\), sometimes denoted \(f^\star
  R\), is the binary relation on \(X\) defined by
  \[(f^\star R)(x, y) \iff R(f(x), f(y)) \text{.}\]
\end{definition}

\begin{exercise}
  Let \(X\) and \(Y\) be sets equipped with binary relations \(R_X\)
  and \(R_Y\), respectively, and let \(f:X \to Y\) be a function.
  Show that \(f\) is a homomorphism if and only if \(R_X = f^\star
  R_Y\).  Note that this exercise shows that the pullback of a
  relation along a function yields the unique relation for which that
  function is a homomorphism.
\end{exercise}

\begin{definition}
  An \defterm{isomorphism} is a homomorphism which is additionally a
  bijection.
\end{definition}

\begin{exercise}
  Show that the inverse of a homomorphism is a homomorphism.
\end{exercise}

\subsection{Congruences}
Congruences generalize the ideas of equivalence relation and quotient
map to sets equipped with binary relations.

\begin{definition}
  Let \(X\) be a set equipped with the binary relation \(R\).  A
  \defterm{congruence} of \(X\) is an equivalence relation \(E\) on
  \(X\) with the property that, whenever \(x_0Ex_1\) and \(y_0Ey_1\),
  \(x_0Ry_0\) if and only if \(x_1Ry_1\).  That is, \(R\) depends only
  on the equivalence class of its arguments.
\end{definition}

\begin{definition}
  Let \(X\) be a set equipped with a binary relation \(R\), and let
  \(E\) be a congruence of \(X\).  Then \(R/E\) is the binary relation
  on \(X/E\) defined by
  \[[x]_E (R/E) [y]_E \iff xRy \text{.}\]
  That is, \(R/E\) holds of two equivalence classes only if it holds
  on the members of those classes.  Note that this definition doesn't
  even make sense unless \(E\) is a congruence.
\end{definition}

\begin{exercise}
  Let \(X\) be a set equipped with a binary relation \(R\).  If \(E\)
  is a congruence, show that the quotient map is a homomorphism (when
  \(X/E\) is considered equipped with the binary relation \(R/E\)).
\end{exercise}

From this point on, we will be less careful about naming each relation
uniquely and explicitly in cases where there is no chance for
confusion.

\subsection{Some Relationships Between Types of Orders}

\begin{definition}
  Let \(X\) be a set equipped with preorer \(\leq\).  Let \(\sim\) to
  be the binary relation on \(X\) defined by
  \[x \sim y \iff x \leq y \text{ and } y \leq x \text{.}\]
\end{definition}

\begin{exercise}
  Show that \(\sim\) is a congruence and that the relation induced on
  \(X/{\sim}\) is a partial order.
\end{exercise}

\begin{exercise}\label{exercise:pullback-order-injection}
  Let \(X\) be a set, \(Y\) be a set equipped with a preorder/partial
  order/total order, and \(f:X \to Y\) be an injection.  Show that the
  pullback of the order along \(f\) yields a preorder/partial
  order/total order on \(X\).
\end{exercise}

\begin{definition}
  Let \(X\) be a set equipped with a partial order.  A \defterm{chain}
  is a subset \(C\) of \(X\) such that the order induced on \(C\) (by
  the pullback along the inclusion map) is a total order.  (Note that
  Exercise~\ref{exercise:pullback-order-injection} already tells us
  that we get a partial order; the only requirement we're adding is
  that we can compare any two elements of \(C\).)
\end{definition}

\section{Cardinality}

\section{Well-Order}

\section{Zorn's Theorem}

\section{Ordinals and Cardinals}

\end{document}
