\documentclass[letterpaper]{article}
\usepackage{amsmath}
\usepackage{amsthm}
\usepackage{amssymb}

\title{Notes on Order, Well-Order, and Cardinality}
\author{Mike Haskel}
\date{Fall, 2014}

\newtheorem{theorem}{Theorem}[section]
\newtheorem{corollary}{Corollary}[theorem]
\newtheorem{proposition}[theorem]{Proposition}
\newtheorem{lemma}[theorem]{Lemma}

\theoremstyle{definition}
\newtheorem{definition}[theorem]{Definition}
\newtheorem{exercise}[theorem]{Exercise}
\newtheorem{example}[theorem]{Example}

\newcommand{\defterm}{\emph}
\renewcommand{\subset}{\subseteq}
\newcommand{\powerset}{\mathcal{P}}
\newcommand{\cartesianprod}{\times}
\newcommand{\inverse}[1]{{#1^{-1}}}
\newcommand{\compose}{\circ}

\begin{document}
\maketitle

\section{Introduction}
The purpose of these notes is to rigorously introduce a handful of
related core concepts that permeate mathematics.  The intended
audience is any undergraduate or accellerated high school student who
is familiar with the basic notation and concepts surrounding sets and
functions.  The student should also be somewhat comfortable with the
language involved in making and proving precise mathematical
statements; they should, for example, be able to explain the
difference between the sentences
\begin{quote}
  for all $x$, there is a $y$ such that $x+y=0$
\end{quote}
and
\begin{quote}
  there is $y$ such that, for all $x$, $x+y=0$.
\end{quote}

\section{Sets, Functions, and Relations}

\begin{definition}
  A \defterm{set} is any collection of elements.  If \(X\) is a set,
  we write \(x \in X\) to denote that \(x\) is an element of \(X\).
  Two sets are considered equal if they have the same elements.
  \footnote{We will sometimes refer to a collection of elements as a
    \defterm{class}.  The distinction between a set and a class is
    subtle and necessary only to avoid certain technical
    contradictions, but for most purposes sets and classes behave
    identically; in particular, almost all definitions and theorems we
    give for sets work just as well for classes.  The main practical
    distinctions between the two concepts are that
    \begin{itemize}
      \item we can usually only form a set using rules based on some
        other set (e.g., if \(X\) is a set, the collection of subsets
        \(X\) is a set), while we can form a class based on more
        arbitrary rules (e.g, the collection of all finite sets is a
        class); and
      \item we allow sets to be elements of other collections, but we
        do not consider collections whose elements are classes.
    \end{itemize}}
\end{definition}

\begin{definition}
  Let \(X\) and \(Y\) be sets.  We say that \(X\) is a
  \defterm{subset} of \(Y\), and write \(X \subset Y\), if every
  element of \(X\) is also an element of \(Y\).
\end{definition}

\begin{definition}
  Let \(x\) and \(y\) be anything.  By \((x,y)\) we denote the
  \defterm{ordered pair} consisting of \(x\) followed by \(y\).  Given
  two ordered pairs \((x_0,y_0)\) and \((x_1,y_1)\), we consider
  \((x_0,y_0) = (x_1,y_1)\) if \(x_0 = x_1\) and \(y_0 = y_1\).
\end{definition}

\begin{definition}
  Let \(X\) and \(Y\) be sets.  The \defterm{Cartesian product} of
  \(X\) and \(Y\), written \(X \cartesianprod Y\), is the set of all
  ordered pairs \((x,y)\) such that \(x \in X\) and \(y \in Y\).
\end{definition}

\begin{definition}
  Let \(X\) and \(Y\) be sets.  A \defterm{binary relation} from \(X\)
  to \(Y\) is a subset of \(X \cartesianprod Y\).  If \(X = Y\), we
  say that we have a binary relation on \(X\).  If \(R\) is a binary
  relation from \(X\) to \(Y\), \(x \in X\), and \(y \in Y\), we write
  either \(R(x,y)\) or \(xRy\) as shorthand for \((x,y) \in R\).
\end{definition}

\begin{definition}
  Let \(X\) be a set.  By the \defterm{power set} of \(X\), denoted
  \(\powerset(X)\), we mean the set of subsets of \(X\).  That is,
  \[A \in \powerset(X) \iff A \subset X \text{.}\]
\end{definition}

\subsection{Relations Between Sets}
\begin{definition}
  Let \(R\) be a binary relation from \(X\) to \(Y\).  \(R\) is
  \defterm{left total} if, for all \(x \in X\), there is at least one
  \(y \in Y\) such that \(xRy\).  \(R\) is \defterm{right total} if,
  for all \(y \in Y\), there is at least one \(x \in X\) such that
  \(xRy\).
\end{definition}

\begin{definition}
  Let \(R\) be a binary relation from \(X\) to \(Y\).  \(R\) is
  \defterm{right unique} if, for all \(x \in X\), there is at most one
  \(y \in Y\) such that \(xRy\).  \(R\) is \defterm{left unique} if,
  for all \(y \in Y\), there is at most one \(x \in X\) such that
  \(xRy\).  (Beware the unfortunately asymmetric use of left and right
  in the definitions of total and unique.)
\end{definition}

If \(y\) is the unique element such that \(x R y\), we often write
\(R(x)\) to denote \(y\).  Similarly, if \(x\) is the unique element
such that \(xRy\), we often write \(\inverse{R}(y)\) to denote \(x\).

\begin{definition}
  A relation which is left total and right unique is called a
  \defterm{function}.  A function which is also left unique is called
  an \defterm{injection} (or is said to be \defterm{injective}).  A
  function which is also right total is called a \defterm{surjection}
  (or is said to be \defterm{surjective}).  A function which is both
  an injection and a surjection (so it is left total, right total,
  left unique, and right unique) is called a \defterm{bijection} (or
  is said to be \defterm{bijective}).
\end{definition}

We often indicate that \(f\) is a function from \(X\) to \(Y\), by
writing \(f:X \to Y\).

\begin{definition}
  Let \(X\) be a set.  By \(id_X\), we mean the binary relation on
  \(X\) defined by
  \[id_X(x,y) \iff x = y \text{.}\]
\end{definition}

\begin{definition}
  Let \(X\) and \(Y\) be sets, and let \(R\) be a binary relation from
  \(X\) to \(Y\).  By \(\inverse{R}\), we mean the binary relation
  from \(Y\) to \(X\) defined by
  \[\inverse{R}(y,x) \iff R(x,y) \text{.}\]
\end{definition}

\begin{definition}
  Let \(X\), \(Y\), and \(Z\) be sets, let \(R\) be a binary relation
  from \(X\) to \(Y\), and let \(S\) be a binary relation from \(Y\)
  to \(Z\).  By \(S \compose R\), we mean the binary relation from
  \(X\) to \(Z\) defined by
  \[S \compose R(x,z) \iff \text{ there is } y \in Y \text{ such that } R(x,y) \text{ and } S(y,z) \text{.}\]
\end{definition}

\begin{exercise}
  Show the following basic facts.
  \begin{enumerate}
    \item \(id_X\) is a bijection.
    \item \(\inverse{(\inverse{R})} = R\).
    \item \(\inverse{R}\) is left/right unique/total if and only if
      \(R\) is right/left unique/total.  Note in particular that the
      inverse of a function is a function if and only if both are
      bijections.
    \item If \(R\) is a binary relation from \(X\) to \(Y\), \(id_Y
      \compose R = R = R \compose id_X\).
    \item Composition is associative: \((R \compose S) \compose T = R
      \compose (S \compose T)\).
    \item \(\inverse{(R \compose S)} = \inverse{S} \compose
      \inverse{R}\).
    \item Let \(R\) be a binary relation from \(X\) to \(Y\).  If
      \(R\) is left total and left unique, then \(\inverse{R} \compose
      R = id_X\).  If \(R\) is right total and right unique, then \(R
      \compose \inverse{R} = id_Y\).
    \item If \(R\) and \(S\) are both left/right total/unique, then
      \(R \compose S\) is left/right total/unique.
  \end{enumerate}
\end{exercise}

\subsection{Relations On Sets}
\begin{definition}
  Let \(R\) be a binary relation on \(X\).  \(R\) is
  \defterm{reflexive} if, for all \(x \in X\), \(xRx\).
\end{definition}

\begin{definition}
  Let \(R\) be a binary relation on \(X\).  \(R\) is
  \defterm{transitive} if whenever \(xRy\) and \(yRz\), we
  additionally have \(xRz\).  When this occurs, we write \(xRyRz\) as
  shorthand.
\end{definition}

\begin{definition}
  Let \(R\) be a binary relation on \(X\).  \(R\) is
  \defterm{symmetric} if whenever \(xRy\), we additionally have
  \(yRx\).
\end{definition}

\begin{definition}
  Let \(R\) be a binary relation on \(X\).  \(R\) is
  \defterm{antisymmetric} if whenever \(xRy\) and \(yRx\), we
  additionally have \(x = y\).
\end{definition}

\begin{definition}
  A preorder on \(X\) is a binary relation on \(X\) which is reflexive
  and transitive.
\end{definition}

\begin{definition}
  An equivalence relation on \(X\) is a preorder on \(X\) which is
  additionally symmetric.
\end{definition}

\begin{definition}
  A partial order on \(X\) is a preorder on \(X\) which is
  additionally antisymmetric.
\end{definition}

\begin{exercise}
  Let \(X\) be a set.  Show that \(id_X\) is an equivalence relation
  on \(X\).
\end{exercise}

\begin{exercise}
  Let \(X\) be a set.  Show that the subset relation is a partial
  order on \(\powerset(X)\).
\end{exercise}

\begin{definition}
  Let \(X\) be a set, and \(E\) be an equivalence relation on \(X\).
  A subset \(A\) of \(X\) is called an \(E\) \defterm{equivalence
    class} if there is some \(a \in A\) such that, for all \(x \in
  X\), \(aEx\) if and only if \(x \in A\).
\end{definition}

\begin{exercise}
  Let \(X\) be a set, \(E\) be an equivalence relation on \(X\), \(A
  \subset X\) be an \(E\) equivalence class, and \(a \in A\).  Then,
  for all \(x \in X\), \(aEx\) if and only if \(x \in A\).  That is,
  for any equivalence class, any element of that class can serve as
  the element used in the definition.
\end{exercise}

\begin{definition}
  Let \(X\) be a set, and \(E\) be an equivalence relation on \(X\).
  The \defterm{quotient} of \(X\) by \(E\), denoted \(X/E\), is the
  set of \(E\) equivalence classes.  That is,
  \[A \in X/E \iff A \text{ is an } E \text{ equivalence class.}\]
\end{definition}

\begin{exercise}
  Let \(X\) be a set, \(E\) be an equivalence relation on \(X\), and
  \(x \in X\).  Show that there is a unique \(E\) equivalence class
  containing \(x\), which we will denote \([x]_E\).  Note that this
  defines an injection from \(X\) to \(X/E\), called the
  \defterm{quotient function}.
\end{exercise}

\section{Orders}

\section{Cardinality}

\section{Well-Orders}

\section{Zorn's Theorem}

\section{Ordinals and Cardinals}

\end{document}
