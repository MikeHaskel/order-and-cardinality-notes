\documentclass[letterpaper]{article}
\usepackage{amsmath}
\usepackage{amsthm}
\usepackage{amssymb}

\title{Notes on Order, Well-Order, and Cardinality}
\author{Mike Haskel}
\date{Fall, 2014}

\newtheorem{theorem}{Theorem}[section]

\theoremstyle{definition}
\newtheorem{definition}[theorem]{Definition}
\newtheorem{exercise}[theorem]{Exercize}
\newtheorem{example}[theorem]{Example}

\newcommand{\integers}{\mathbb{Z}}
\newcommand{\nnintegers}{{\integers^{\geq 0}}}

\begin{document}
\maketitle

\section{Introduction}
The purpose of these notes is to rigorously introduce a handful of
related core concepts that permeate mathematics.  The intended
audience is any undergraduate or accellerated high school student who
is familiar with the basic notation and concepts surrounding sets and
functions.  The student should also be somewhat comfortable with the
language involved in making and proving precise mathematical
statements; they should, for example, be able to explain the
difference between the sentences
\begin{quote}
  for all $x$, there is a $y$ such that $x+y=0$
\end{quote}
and
\begin{quote}
  there is $y$ such that, for all $x$, $x+y=0$.
\end{quote}

These notes will introduce the concepts of order, well-order, and
cardinality.  Order is what it sounds like: any way of comparing two
things that behaves as a comparison of size.  We draw attention to
some of the different kinds of orders and the axioms (rules) they
obey.  A particularly important type of order is a \emph{well-order}.
Well orders behave as the natural numbers do in that we can perform a
certain kind of induction them.  We then use the ideas surrounding
well-orders to prove \emph{Zorn's Theorem}, a broad and important
mathematical tool.  With all these concepts in our arsenal, we finally
turn to introduce the idea of \emph{cardinality}: the size of a
(possibly infinite) set.

\section{Background Review: Sets, Relations, and Functions}
TODO.

\section{Order}
We want to study the idea of an order: a binary relation that behaves
as a comparison of size.  We approach this idea in three steps, each
of which is a useful concept in its own right.

\begin{definition}
  \label{def:preorder}
  Let \(X\) be a set and \(\leq\) a binary relation on \(X\).  Then
  \(\leq\) is a \emph{preorder} on \(X\) if it is reflexive and
  transitive.
\end{definition}

\begin{definition}
  \label{def:partialorder}
  Let \(X\) be a set and \(\leq\) a preorder on \(X\).  Then \(\leq\)
  is a \emph{partial order} if it is also antisymmetric.
\end{definition}

Note that the only defect of a preorder that prevents it from being a
partial order is that it may be possible to have two elements \(a,b\)
in \(X\) with \(a \leq b\) and \(b \leq a\), but nevertheless \(a \neq
b\).  When \(\leq\) is a partial order, it is common to write \(a \sim
b\) to denote that \(a \leq b\) and \(b \leq a\).  We can also use \(a
\geq b\) as obvious shorthand for \(b \leq a\), and \(a < b\) as
shorthand for indicating simultaneously that \(a \leq b\) and \(a
\not\sim b\).

\begin{example}
  \label{example:prenotpartial}
  Consider the set
  \[X = \{\text{black}, \text{red}, \text{blue}\}\text{,}\]
  and the binary relation
  \begin{align*}
    \mathord{\leq} = \{
    &(\text{black}, \text{black}), \\
    &(\text{black}, \text{red}), \\
    &(\text{black}, \text{blue}), \\
    &(\text{red}, \text{red}), \\
    &(\text{red}, \text{blue}), \\
    &(\text{blue}, \text{red}), \\
    &(\text{blue}, \text{blue})
    \} \text{.}
  \end{align*}
  It is routine to check that \(\leq\) is a preorder.  It is not a
  partial order, however, because we have \(\text{red} \sim
  \text{blue}\); that is, \(\text{red} \leq \text{blue}\) and
  \(\text{blue} \leq \text{red}\).
\end{example}

\begin{example}
  \label{example:partialnotlinear}
  Consider the set
  \[X = \{\text{black}, \text{red}, \text{blue}\}\text{,}\]
  and the binary relation
  \begin{align*}
    \mathord{\leq} = \{
    &(\text{black}, \text{black}), \\
    &(\text{black}, \text{red}), \\
    &(\text{black}, \text{blue}), \\
    &(\text{red}, \text{red}), \\
    &(\text{blue}, \text{blue})
    \} \text{.}
  \end{align*}
  As in Example~\ref{example:prenotpartial}, it is routine to check
  that \(\leq\) is a preorder.  With our modification, however,
  \(\leq\) is also a partial order.  Note in particular the new
  behavior of red and blue: neither one is \(\leq\) the other.  We say
  that red and blue are \emph{incomparable}.
\end{example}

Our next concept is that of a \emph{total} (sometimes called
\emph{linear}) order.

\begin{definition}
  Let \(X\) be a set, and let \(\leq\) be a preorder on \(X\).  If
  there are no incomparable elements (that is, for all \(a\) and \(b\)
  in \(X\), either \(a \leq b\) or \(b \leq a\)), then we call
  \(\leq\) a \emph{total preorder}.  If \(\leq\) is additionally a
  partial order, we just call \(\leq\) a \emph{total order}.
\end{definition}

The usual orders on the real numbers and on the integers are examples
of total orders.

\subsection{Chains and Orders on Subsets}
TODO

\subsection{Well-Order, Induction, and Recursion}
Consider two examples of total orders: the real numbers and the
nonnegative integers, each with their usual orders.  There are many
differences between these orders, but one key difference is that it is
possible to define a function all the positive integers by simply
specifying how the value depends on the values the function takes on
lesser integers.  For example, consider the function \(f:\nnintegers
\to \integers\) defined by
\[ f(n) = 1+\sum_{k<n} f(k) \text{.} \]
We can make sense of this definition: \(f(0) = 1\) (since we interpret
an empty sum as having value 0), \(f(1) = 1+1 = 2\), \(f(3) = 1+1+2 =
4\), etc.  It is not possible to make sense of a definition like this
for the real numbers without resorting to calculus, but calculus
relies on properties beyond just the order itself.

The key property of the nonnegative integers that allows us to make
sense of the preceding definition is that the nonnegative integers
have are \emph{well-ordered}.
\begin{definition}
  Let \(X\) be a set, and let \(\leq\) be a total order on \(X\).  If
  every nonempty subset of \(X\) has a least element (that is, an
  element of the subset \(\leq\) all other elements of the subset),
  then we say that \(\leq\) is a \emph{well-order}.
\end{definition}

This property is equivalent to two other important properties.
\begin{definition}
  Let \(X\) be a set, and let \(\leq\) be a total order on \(X\).  Let
  \(P \subset X\), and think of \(P\) as a unary relation: that is, a
  property that may or may not hold of any individual element of
  \(X\).  Assume we know that, whenever \(x \in X\) is such that \(P\)
  holds of every \(y < x\), \(P\) in fact holds of \(x\).  In this
  case, we wish to conclude that \(P = X\), i.e.\ \(P\) holds of every
  \(x \in X\).  If this is true of \emph{every} \(P\) satisfying the
  stated assumption, then we say that \(\leq\) has the
  \emph{transfinite induction property}.
\end{definition}
\begin{definition}
  Let \(X\) be a set, and let \(\leq\) be a total order on \(X\).  We
  say that \(\leq\) \emph{admits an infinite descending sequence} if
  it is possible to find \(x_0, x_1, \ldots\) such that
  \[x_0 > x_1 > x_2 > \ldots\]
\end{definition}
\begin{theorem}
  Let \(X\) be a set, and let \(\leq\) be a total order on \(X\).
  Then the following are equivalent:
  \begin{enumerate}
    \item \(\leq\) is a well-order,
    \item \(\leq\) has the transfinite induction property, and
    \item \(\leq\) does not admit an infinite descending sequence.
  \end{enumerate}
\end{theorem}

\end{document}
