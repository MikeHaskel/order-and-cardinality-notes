\documentclass[letterpaper]{article}
\usepackage{amsmath}
\usepackage{amsthm}
\usepackage{amssymb}

\title{Notes on Order, Well-Order, and Cardinality}
\author{Mike Haskel}
\date{Fall, 2014}

\newtheorem{theorem}{Theorem}[section]

\theoremstyle{definition}
\newtheorem{definition}[theorem]{Definition}
\newtheorem{exercise}[theorem]{Exercize}
\newtheorem{example}[theorem]{Example}

\begin{document}
\maketitle

\section{Introduction}
The purpose of these notes is to rigorously introduce a handful of
related core concepts that permeate mathematics.  The intended
audience is any undergraduate or accellerated high school student who
is familiar with the basic notation and concepts surrounding sets and
functions.  The student should also be somewhat comfortable with the
language involved in making and proving precise mathematical
statements; they should, for example, be able to explain the
difference between the sentences
\begin{quote}
  for all $x$, there is a $y$ such that $x+y=0$
\end{quote}
and
\begin{quote}
  there is $y$ such that, for all $x$, $x+y=0$.
\end{quote}

\end{document}
